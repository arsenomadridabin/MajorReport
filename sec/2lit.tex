\section{LITERATURE REVIEW}
\subsection{Theory Details}
When predicting the future prices of Stock Market securities, there are several theories available. The first is Efficient Market Hypothesis (EMH). In EMH, it is assumed that the price of a security reflects all of the information available and that everyone has some degree of access to the information. Fama\"s theory further breaks EMH into three forms: Weak, Semi-strong, and Strong. In Weak EMH, only historical information is embedded in the current price. The Semi-Strong form goes a step further by incorporating all historical and currently public information in the price. The Strong form includes historical, public, and private information, such as insider information, in the share price. From the tenets of EMH, it is believed that the market reacts instantaneously to any given news and that it is impossible to consistently outperform the market.

 A different perspective on prediction comes from Random Walk Theory. In this theory, Stock Market prediction is believed to be impossible where prices are determined randomly and outperforming the market is infeasible. Random Walk Theory has similar theoretical underpinnings to Semi-Strong EMH where all public information is assumed to be available to everyone. However, Random Walk Theory declares that even with such information, future prediction is ineffective.

 It is from these theories that two distinct trading philosophies emerged; the fundamentalists and the technicians. In a fundamentalist trading philosophy, the price of a security can be determined through the nuts and bolts of financial numbers. These numbers are derived from the overall economy, the particular industry\'s sector, or most typically, from the company itself. Figures such as inflation, joblessness, return on equity (ROE), debt levels, and individual Price to Earnings (PE) ratios can all play a part in determining the price of a stock.

 In contrast, technical analysis depends on historical and time-series data. These strategists believe that market timing is critical and opportunities can be found through the careful averaging of historical price and volume movements and comparing them against current prices. Technicians also believe that there are certain high/low psychological price barriers such as support and resistance levels where opportunities may exist. They further reason that price movements are not totally random, however, technical analysis is considered to be more of an art form rather than a science and is subject to interpretation.

Both fundamentalists and technicians have developed certain techniques to predict prices from financial news articles. In one model that tested trading philosophies, LeBaron et. al. posited that when predicting the future prices of Stock Market securities, there are several theories available. The first is Efficient Market Hypothesis (EMH). In EMH, it is assumed that the price of a security reflects all of the information available and that everyone has some degree of access to the information.

In similar research on real stock data and financial news articles, Gidofalvi gathered over 5,000 financial news articles concerning 12 stocks, and identified this brief duration of time to be a period of twenty minutes before and twenty minutes after a financial news article was released. Within this period of time, Gidofalvi demonstrated that there exists a weak ability to predict the direction of a security before the market corrects itself to equilibrium. One reason for the weak ability to forecast is because financial news articles are typically reprinted throughout the various news wire services. Gidofalvi posits that a stronger predictive ability may exist in isolating the first release of an article. Using this twenty-minute window of opportunity and an automated textual news parsing system, the possibility exists to capitalize on stock price movements before human traders can act. 

Also Ralph Nelson Elliott developed the Elliott wave theory in the late 1920s by discovering that stock markets, thought to behave in a somewhat chaotic manner, in fact traded in repetitive cycles. Elliott discovered that these market cycles resulted from investors reactions to outside influences, or predominant of psychology of the masses at the time. He found that the upward and downward swings of the mass psychology always showed up in the same repetitive patterns, which were then divided further into patterns he termed \textbf{waves}.

For the stock market prediction, Artificial Neural Network(ANN) has been considered the most efficient method. Inspired by neurosciences, ANNs have shown great potential in terms of recognizing patterns in nonlinear systems. Existing research suggests that ANN is an eminent model to predicting stock markets due to its dynamical characteristics. Even so, a common criticism of neural networks is that they require a large diversity of training for real-world operation. Moving average analysis and single exponential smoothing methods are frequently used in order to make stock analysis. The Nepal stock exchange (NEPSE) uses exponential smoothing in its website for this purpose. Moving averages work quite well in strong trending conditions, but often poorly in choppy or ranging conditions.

Under the assumption that the stock market could be predicted, there are some major cateogories of prediction methods: fundamental analysis, technical analysis and news analysis.\\

	\textbf {I. Fundamental Analysis}\\
	It mainly depends on statistical data of a company. It includes reports, financial status of the company, the balance sheets, dividends and policies of the companies whose stock are to be observed. It also includes analysis of market data, strength and investment of company, the competition, import/export volume, production indices, price statistics of the company.

	\textbf{II. Technical Analysis}\\
	In stock analysis there are two approaches, first approach includes analysis of graphs where analysts try to find out certain patterns that are followed by stock but this approach is very difficult and complex to be used with ANN. In second approach analysts make use of quantitative parameters like trend indicators, daily ups and downs, highest and lowest values of a day, volume of stock, indices, pull/call ratios, etc. It also includes some averages which is nothing more than mean of prices for particular window size like Simple Moving Average(MA) and Exponential Moving Average(EMA). Here prices of recent days have more weight in average. Analysts try to find out some mathematical formula which can map this input in the desired output.

	\textbf{III. News Analysis}\\
	Some of the researchers showed that there is a strong relationship between news article about a company and its stock prices fluctuations. This analysis includes the processing of analyzing the news of the company using Natural Language Processing techniques.

Both long-term and short-term stock price can be calculated considering the above analysis. Technical and fundamental analysis could be adopted for long term prediction whereas news sentiment analysis could play a handy role for short-term predicitons.

\subsection{Related Work}
Many algorithms of data mining have been proposed to predict stock price. Neural Network, Genetic Algorithm, Decision Tree and Fuzzy systems are widely used. In addition, pattern discovery is beneficial for stock market prediction and public sentiment is also related to predicting stock price.

There are a lot of software and web applications working with the similar concept. Nepal Sharemarket is a website that makes individual, comparative as well as in depth analysis on stock market companies and also forecasts their price on a chosen time basis.

Another website, Stock-forecasting.com also makes stock prediction using neural networks and boasts of highest accuracy among all the stock-prediction applications. It is an American company and gives minute predictions of various international companies.

Other software like InteliCharts and Addaptron also make stock predictions based on neural networks.